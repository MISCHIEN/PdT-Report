\documentclass{report}

\usepackage[utf8]{inputenc}
\usepackage[T1]{fontenc}
\usepackage[francais]{babel}
\usepackage{graphicx}
\usepackage{hyperref}
\usepackage{textcomp}
\usepackage{amsmath}
\usepackage{geometry}
\usepackage{pdfpages}

\usepackage{array,multirow,makecell}
\setcounter{secnumdepth}{3}
\setcounter{tocdepth}{3}
\setcellgapes{4pt}
\makegapedcells
\newcolumntype{R}[1]{>{\raggedleft\arraybackslash }b{#1}}
\newcolumntype{L}[1]{>{\raggedright\arraybackslash }b{#1}}
\newcolumntype{C}[1]{>{\centering\arraybackslash }b{#1}}

\usepackage[pages=some]{background}
\backgroundsetup{
	scale=1,
	color=black,
	opacity=0.1,
	angle=0,
	contents={%
		\includegraphics[width=\paperwidth,height=\paperheight]{img/ulbback.jpg}
	}%
}


\title{\Huge\emph{Cooling down a coke can}\\
	\LARGE Experiment study\\
	\vspace{11pt}
	\normalsize Fluid Mechanics and Transport Processes}

\date{December 2015}

\author{Nathan DE PRYCK \and Miquel KEGELEIRS \and Mischa MASSON}



\begin{document}
	
	\begin{figure}[t]
		\includegraphics[width=15cm]{img/entete.PNG}
	\end{figure}
	
	\maketitle
	
	\renewcommand{\abstractname}{``Cooling down a coke can'' \\Experiment study by Nathan De Pryck, Miquel Kegeleirs and Mischa Masson\\ Université Libre de Bruxelles\\2015-2016.}
	
	\BgThispage
	
	\begin{abstract}
		TO BE WRITTEN
	\end{abstract}
	
	\clearpage
	
	
	\tableofcontents
	
	\chapter{Introduction}\label{intro}
	
	This report outlines the physics behind an experiment produced during the MECA-H3001 ``Fluid Mechanics and Transport Processes'' course of the ULB (Université Libre de Bruxelles) on Friday, the sixteenth of October 2015.
	
	The experiment can be found by following \hyperref{https://www.youtube.com/watch?v=MSwc_IAPh3E}{}{}{this link}\footnote{\url{https://www.youtube.com/watch?v=MSwc_IAPh3E}} and was described as following:
	
	\begin{itemize}
		\item Three coke cans were available as well as a bucket of ice and water (at 0\textcelsius) and a drill to spin the can inside the bucket.
		\item One of the cans was used to determine the initial temperature (16\textcelsius) of the fluid (essentially water) inside the can.
		\item One can was left inside the bucket for 60 seconds and a final temperature of 11.9\textcelsius \ was measured as a result of the conductive heat transfer with the surrounding fluid.
		\item One can was spun inside the bucket for 60 seconds at about 1000 rounds per minute, and a final temperature of 11\textcelsius \ was measured as a result of the convective and conductive heat transfer.
	\end{itemize}
	
	In the first chapter, convective and conductive processes that are related to the experiment will be described. Those phenomena will then be applied to the problem via a mathematical model, including a discussion of simplifications brought to the problem in order to simplify calculations. Lastly, a comparison between the model and reality followed by a conclusion on mathematical models will be presented.
	
	\chapter{Heat transfer processes}\label{htp}
	
	They are three different ways of transferring heat: conduction, convection and radiation. Due to its nature, radiation can be neglected for this experiment and will thus not be presented here.
	
	\section{Conduction}\label{cd}
	
	Thermal conduction is a heat transfer process without macroscopic movement of matter. It is initiated by a difference of temperature between contiguous bodies (or inside a body). This difference of temperature implies a difference of internal energy : the energy is higher in the warmer area than in the cooler. By diffusion and collisions between the particles which can be molecules in a fluid or conduction electrons in a solid, particles in the warmer area transfer kinetic energy to the other particles, making them moving or vibrating faster. This creates a heat flow from the warmer area to the cooler until the system reaches thermal equilibrium. Furthermore, conduction is an irreversible process.
	
	Conduction is described by the following general equation, which is demonstrated in Professor Jean-Marie Buchlin's course\cite{Buchlin}, Chapter 13.
	
	\begin{equation}
		\frac{\delta T}{\delta t} = \bigtriangledown . (\alpha \bigtriangledown T)+ \dot{Q}_v
	\end{equation}
	
	This equation can not be used by itself because of it's nature (second degree partial derivative equation). It thus needs conditions linked to properties of the system. Those can be geometrical, physical, temporal or border conditions.
	
	Conditions used and simplifications of the general equation above will be discussed in chapter~\ref{mm}
		
	\section{Convection}\label{cv}
	
	Convection is a heat transfer in fluid. Convection occurs when some fluid is in movement. The movement lead to an advection (heat is transported by matters when it's moving).
	
	Convection is described as following :
	\begin{equation}
		Convection = Conduction + Advection
	\end{equation}
	
	Seeing this, it is easily to understand that convection is superior than conduction in fluids in a flux situation. Flow properties have a major impact in heat transfers.
	
	As convection depends on the flow (laminar, turbulent,...), we will discuss the equation to use in the next chapter(Mathematical model,chapter~\ref{mm}).
	
	\chapter{Mathematical model}\label{mm}
	
	The idea behind mathematical models is to create a simplified version of a problem, that is accurate enough to predict the behaviour of a system, but simple enough to be resolved with few calculations.
	
	This means that some simplifications of the equations seen before can be made, using the properties of the studied system.
	
	We will first describe general simplifying assumptions for our experiment and explain why we can use them, after what we will go ahead and create two simplified models: one for the non-spinning can and one for the spinning can.
	
	\section{General simplifying assumptions}\label{gsa}
	
	The first assumption we will consider is that the fluid contained in the can has properties similar to water. Coke is indeed an aqueous solution containing sugar and other ingredients, but at relatively low concentrations.
	Properties of water can be found in annex~\ref{WTPP} and are extracted from \emph{``Perry’s Chemical Engineers’ Handbook''\cite{properties}}.
	
	Another assumption is that the can is a perfect cylinder with an height of $h=116mm$ and a diameter of $d=66mm$\footnote{dimensions are standard ones for European Aluminium $330ml$ cans, those can be found on \hyperref{http://www.webpackaging.com/en/portals/rexam/assets/11059498/spec-alu-202/}{}{}{webpackaging.com}}. In the reality, the shape of a can is a bit different to support pressure but difference should not be significant in our calculations. The material used for cans is Aluminium and since it is extremely thin(less than $1mm$) and has an excellent thermal conductivity ($>200\frac{W}{K.m}$), we will consider it has no influence on heat transfers. We will thus consider a cylinder of water inside water maintained at another temperature and without any matter exchange, as shown in figure~\ref{cyl}.
	
	\begin{figure}
		\label{cyl}
		\centering
		\includegraphics[width=.5\textwidth]{img/cyl.jpg}
		\caption{The red cylinder is hotter water, the blue box is colder water. No matter exchanges, only heat.}
	\end{figure}
	
	Furthermore, we will consider that all heat exchanges between the can and surrounding ice-cold water takes place on de sides of the can and not on it's top or bottom. The total surface of the can is given by:
	
	\begin{equation}
	\begin{gathered}
		Surface= 2.Surface_{circle} + Surface_{rectangular side} \\
		\Leftrightarrow S= 2 (\pi  (\frac{d}{2})^2) + \pi dh \\
		\Leftrightarrow S= 30.8944 cm^3
	\end{gathered}
	\end{equation}
	
	Bottom and top circular surfaces have a total surface of $2xSurface_{circle}=2 (\pi  (\frac{d}{2})^2)= 6.8424 cm^3$, this is thus about $22.14\% $ of total surface. Reason why we decided to ignore such a large portion of the can's area is because half of it is not even in contact with ice during the experiment (the system holding the can covers it's top and non-spinning can is not totally in ice), reducing the area to consider at about $11\% $ of total exchange area. This can still be seen as a large percentage, but because it is on the bottom of the can and because we are trying to cool down a liquid and cold fluids tend to be more dense than hot fluid, the liquid in contact with the bottom part will be cooler, making heat transfer far less efficient there.
	
	The last general simplification we will make is to consider the mix of ice and water surrounding the cans as a continuous layer of water maintained at 0\textcelsius. This should be correct enough considering that we had a large quantity of melting ice in an isolated box and because liquid water is more flexible than ice, the can has a larger surface in contact with molten ice (and thus water) than with ice itself.
	
	
	\section{Simplified models}\label{sm}
	
	With these four simplifications in mind, we will now build two models, one for the non-spinning can and one for the spinning can.
	
	\subsection{Non-spinning can}\label{nrc}
	
	Heat transfer processes in presence are conduction and natural convection. Natural convection appears because of the density change of fluid related to temperature: the cooler the fluid, the denser it is. There is no forced convection because there is no flux in the fluid.
	
	The convective heat transfer coefficient for natural convection (in the can) is given by:
	\begin{equation}
		h_x=Ra^\alpha = cGr^\alpha Pr^\alpha
	\end{equation}
	
	Where $Gr$ is the Grashof Number and $Pr$ the Prandtl number. Prandtl is found in annex~\ref{WTPP}: water being at 16\textcelsius (about $290K$), we have $Pr=7.56$.\\
	Grashof number is calculated using the following formula:
	\begin{equation}
	Gr=\frac{\beta g \Delta T_{ref} x^3}{\nu^2}
	\end{equation}
	
	Where:
	\begin{itemize}
		\item $\beta$ is the volume expansion coefficient, given by $\frac{1}{\rho}\frac{\delta\rho}{\delta T}$. Using the tables in annex~\ref{WTPP}, we can approximate $\beta=1.001\times 10^{-3}\frac{\frac{1}{1.000\times 10^{-3}}-\frac{1}{1.001\times 10^{-3}}}{5}=2\times 10^{-4} K^{-1}$.
		\item $g=9.81\frac{m^2}{s}$ is gravity.
		\item $\Delta T_{ref}=16-0=16$\textcelsius\ is the difference of temperature between the water inside the can and the ice-cold water outside of it.
		\item $x=116\times 10^{-3}m$ is the height of the cylinder.
		\item $\nu=\frac{\mu}{\rho}=\frac{1080\times 10^{-6}}{\frac{1}{1.001\times 10^{-3}}}=1.081\times 10^{-6}\frac{m^2}{s}$ is water's kinematic viscosity.
	\end{itemize}
	All values used are from annex~\ref{WTPP}.
	
	
	Rayleigh is thus: 
	\begin{equation}
		Ra = GrPr= 7.56\times \frac{1.998\times 10^{-4}\times 9.81\times 16\times (116\times 10^{-3})^3}{(1.081\times 10^{-6})^2}= 3.1669\times 10^{8}
	\end{equation}
	Since $Ra<10^9$, we are in a laminar case. We consider that the ice-cold water around the can is at a constant temperature, we will use the formula for ??? given in annex~\ref{FORMU}:
	\begin{equation}
		\begin{gathered}
		Nu=???\\
		\Leftrightarrow Nu=XXX
		\end{gathered}	
	\end{equation}
	and:
	\begin{equation}
		\begin{gathered}
		Nu=\frac{hd}{k}\\
		\Leftrightarrow h=\frac{Nu k}{d}\\
		\Leftrightarrow h=XXX \frac{W}{m^2K}
		\end{gathered}	
	\end{equation}
	
	Biot number is given by $Bi=XXX$. To simplify the equation, we will use the equations for a Lumped system in the case of sensible heat transfer:
	
	\begin{equation}
		\begin{gathered}
		\rho CV\frac{dT}{dt}=-hS_{tot}(T-T_{ext})\\
		\Leftrightarrow \frac{hS_{tot}}{\rho CV}dt=\frac{1}{(T_{ext}-T)}dT\\
		\Leftrightarrow \frac{4h}{\rho Cd}dt=\frac{1}{(T_{ext}-T)}dT\\
		\Leftrightarrow \frac{4h}{\rho Cd}t=ln(\frac{T_0-T_{ext}}{T_f-T_{ext}})\\
		\Leftrightarrow e^{\frac{4h}{\rho Cd}t}=\frac{T_0-T_{ext}}{T_f-T_{ext}}\\
		\Leftrightarrow T_f=T_0+\frac{T_0-T_{ext}}{e^{\frac{4h}{\rho Cd}t}}
		\end{gathered}
	\end{equation}
	
	This, when we use the parameters of our experiment, would have given $T_f=XXXK=YYY$\textcelsius\ after $t=60s$.
	
	\subsection{Spinning can}
	
	\chapter[Reality and mathematical model]{Comparison between reality and the mathematical model}\label{rvmm}
	
	\chapter{Conclusion}\label{ccl}
	
	
	\appendix
	
	\chapter{Water Thermo-Physical Properties}\label{WTPP}
	
	These chart can be are extracted from ``Perry’s	Chemical Engineers’ Handbook'', Chapter 2\cite{properties}.
	
	\includepdf[pages={-}]{WaterThermo-PhysicalProperties.pdf}
	
	\chapter{Formula Sheet}\label{FORMU}
	
	The following formula sheet was given and demonstrated at Pr. Parente's course ``MECA-H3001: Fluid mechanics and transfer processes''.
	
	\includepdf[pages={-}]{FormulaSheet.pdf}
	
	
	\bibliography{biblio}{}
	\bibliographystyle{plain}
	
\end{document}